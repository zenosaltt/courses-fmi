\chptr{Simbologia}

\begin{align*}
    &\textbf{Simbolo} &\textbf{Significato}                     &&\textbf{Esempio}\\
    &\in              &\text{Appartiene a}                      &&x \in A\\
    &\ni              &\text{Contiene}                          &&A \ni x\\
    &\forall          &\text{per ogni}                          &&\forall x, x = x\\
    &\exists          &\text{esiste (almeno un)}                &&\exists n\in\mathbb{N}: n = 0\\
    &\varnothing      &\text{Insieme vuoto}                     &&\\
    &\subset\text{\footnotemark} &\text{Sottoinsieme di}        && A \subset B\\
    &\subseteq        &\text{Sottoinsieme di}                   &&\\
    &\subsetneq       &\text{Sottoinsieme proprio di}           &&\\
    &\subsetneqq      &\text{Sottoinsieme proprio di}           &&\\
    &:                &\text{Tale che (affermazioni)}           && \exists k\in\mathbb{N}:k=0\\
    &\text{ }|        &\text{Tale che (assioma di separazione)} && \{n|n\in\mathbb{N}, n \text{ è pari}\}
\end{align*}

\footnotetext{La letteratura è ambigua riguardo il significato preciso di questo simbolo. Si consiglia l'utilizzo di altre notazioni più esplicite.}