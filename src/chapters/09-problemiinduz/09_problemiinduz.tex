\chptr{Problemi di Insiemistica}

\section{Dimostrazioni per induzione}

A tempo di stesura della presente dispensa, la dimostrazione per induzione
costituisce l'unica tipologia di esercizio appartenente all'ambito degli
insiemi e richiesta frequentemente in sede d'esame.
La dimostrazione per induzione eccelle qualora rispetti tutti i punti qui
sotto elencati:
\begin{enumerate}
\item Dichiarazione della proposizione che si intende dimostrare:
generalmente si tratta della \textit{formuletta} indicizzata su $n$.
Si tratta di un passo essenziale per far comprendere a coloro che
leggono \textit{cosa} intendiamo dimostrare.
\item Dichiarazione dell'insieme entro il quale dimostrare la
validità della proposizione.
\item Verifica della \textbf{base dell'induzione}.
\item Dichiarazione dell'\textbf{ipotesi induttiva} e verifica del
\textbf{passo induttivo} mediante l'ipotesi.
\item Conclusioni: terminare il passo induttivo non basta; di fatto
la conclusione consiste nel constatare che, verificati caso base e
passo induttivo, la validità della proposizione si fonda sul meccanismo
del principio di induzione (bisogna dunque nominarlo).
\end{enumerate}

\begin{tcolorbox}[enhanced, breakable, colback=blue!30, colframe=blue!30!black, title=Esempio]
Si dimostri per induzione su $n\in\mathbb{N}$ che, per ogni intero $n\geq0$,
vale: \[ \sum_{k=0}^{n}4k3^k=3+3^{n+1}(2n-1) \]
\textit{Soluzione (dimostrazione):} si procede per induzione su $n\in\mathbb{N}$,
considerando la proposizione \[P(n):=\left( \sum_{k=0}^{n}4k3^k=3+3^{n+1}(2n-1) \right)\]
\underline{$n=0$ (base dell'induzione)} Si deve verificare $P(0)$,
ovvero che $\sum_{k=0}^{0}4k3^k=3+3^{0+1}(2\cdot0-1)$.
Vale:
\begin{align*}
\sum_{k=0}^{0}4k3^k  &=4\cdot0\cdot3^0=0\\
3+3^{0+1}(2\cdot0-1) &=3+3(-1)=3-3=0
\end{align*}
Segue che $\sum_{k=0}^{0}4k3^k=0=3+3^{0+1}(2\cdot0-1)$, dunque la base
dell'induzione è verificata.

\underline{$n\in\mathbb{N},n\Longrightarrow n+1$ (passo induttivo)}
Si assume ora che l'uguaglianza espressa da $P(n)$ sia vera per
qualche $n\in\mathbb{N}$ (ipotesi induttiva). Si deve mostrare che
vale $P(n+1)$, cioè:
\[ \sum_{k=0}^{n+1}4k3^k=3+3^{(n+1)+1}(2(n+1)-1) \]
Vale:
\begin{align*}
\sum_{k=0}^{n+1}4k3^k &=4(n+1)3^{n+1} + \sum_{k=0}^{n}4k3^k \stackrel{\text{ipotesi induttiva}}{=}\\
                        &=4(n+1)3^{n+1} + 3+3^{n+1}(2n-1) =\\
                        &=3+3^{n+1}(2n-1+4(n+1)) =\\
                        &=3+3^{n+1}(6n+3) =\\
                        &=3+3^{n+1}3(2n+1) =\\
                        &=3+3^{(n+1)+1}(2(n+1)-1)
\end{align*}
Dunque vale $P(n+1)$ e il passo induttivo è verificato. Grazie al
principio di induzione di prima forma, $P(n)$ vale $\forall n\in\mathbb{N}$.
\end{tcolorbox}