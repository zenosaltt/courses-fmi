\chptr{Congruenza e Aritmetica Modulare}
\section{Congruenza in modulo}
\section{Quoziente insiemistico}
\section{Classi di congruenza}
\section{Struttura algebrica di $\mathbb{Z}/_{n\mathbb{Z}}$}

\section{\underline{Teorema cinese del resto}}
\begin{tcolorbox}[enhanced, breakable, title={Teolema cinese del lesto}]
Siano $n,m>0$ e siano $a,b \in \mathbb{Z}$. Consideriamo il seguente problema:
\[
    \begin{cases}
        x \in \mathbb{Z}\\
        x \equiv a \Mod{n}\\
        x \equiv b \Mod{m}
    \end{cases}
    \Longleftrightarrow
    \begin{cases}
        x \in \mathbb{Z}\\
        [x]_n = [a]_n\\
        [x]_m = [b]_m
    \end{cases}
\]
Indichiamo con $S$ l'insieme delle soluzioni del precedente sistema di congruenze:
$S:=\{ x \in \mathbb{Z}| x \equiv a \Mod{n}, x \equiv b \Mod{m} \}$.
Allora il sistema è compatibile se e solo se $a \equiv b \Mod{(n,m)}$, ovvero:
\[ S\not = \varnothing \Longleftrightarrow (n,m)|a-b \]
Assumiamo che $S \not = \varnothing$ e sia $c\in S$. Allora:
\begin{align*}
    S &=[c]_{[n,m]} \subseteq \mathbb{Z}\\
    &=\{ c+k[n,m] \in \mathbb{Z}| k \in \mathbb{Z} \}
\end{align*}

\emph{\textbf{Dimostrazione:}} \underline{\emph{Compatibilità:}} Assumiamo che
$S \not = \varnothing \Rightarrow \exists c \in S$ ovvero:
\[
    \begin{cases}
        c \equiv a \Mod{n}\\
        c \equiv b \Mod{m}
    \end{cases}
    \Longleftrightarrow
    \begin{cases}
        c = a+kn\\
        c = b+hm
    \end{cases}
\]
per qualche $h,k \in \mathbb{Z}$. Vale: $ 0 = a+kn - b-hm \Leftrightarrow -kn+hm = a-b $.
Ma dato che $(n,m)|n$ e $(n,m)|m$, per il lemma utile $(n,m)|-kn+hm = a-b$.

Assumiamo ora che valga $(n,m)|a-b$. Vale, per qualche $k \in \mathbb{Z}$:
\begin{align}
    a-b = k(n,m)
\end{align}
Applichiamo l'algoritmo di euclide con sostituzione a ritroso ad $n$ e $m$, ottenendo la
combinazione lineare:
\begin{align}
    (n,m) = xn+ym
\end{align}
per qualche $x,y \in \mathbb{Z}$. Grazie a (1) e (2) vale:
\begin{align*}
    &a-b=k(n,m)=k(xn+ym)=kxn+kym \Leftrightarrow\\
    \Leftrightarrow &a+(-kx)n=b+(ky)m =: c \in S
\end{align*}

\underline{\emph{Soluzione:}} Sia $c \in S$ e proviamo che $S=[c]_{[n,m]} \subseteq \mathbb{Z}$.
Mostriamo che $S \subseteq [c]_{[n,m]}$. Sia $c'\in S$. Vale:
\[ c \in S \Longleftrightarrow
    \begin{cases}
        c \equiv a \Mod{n}\\
        c \equiv b \Mod{m}
    \end{cases}
    \Longleftrightarrow
    \begin{cases}
        c = a+kn\\
        c = b+hm
    \end{cases}
\]
\[
    c' \in S \Longleftrightarrow
    \begin{cases}
        c' \equiv a \Mod{n}\\
        c' \equiv b \Mod{m}
    \end{cases}
    \Longleftrightarrow
    \begin{cases}
        c' = a+k'n\\
        c' = b+h'm
    \end{cases}
\]
per qualche $h,h',k,k' \in \mathbb{Z}$. Vale:
\[ c'-c=k'n-kn=(k'-k)n \quad\text{ e }\quad c'-c=h'm-hm=(h'-h)m \]
dunque
\[ n|c'-c,m|c'-c \Longrightarrow [n,m]|c'-c \Longleftrightarrow c'\equiv c \Mod{[n,m]} \Longleftrightarrow c' \in [c]_{[n,m]} \]
Mostriamo che $[c]_{[n,m]} \subseteq S$. Sia $c' \in [c]_{[n,m]}$, ovvero $c'=c+k[n,m]$ per qualche
$k \in \mathbb{Z}$. Vale:
\begin{align*}
    [c']_n &=\\
    &= [c+k[n,m]]_n =\\
    &= [c]_n + [k[n,m]]_n =\\
    &= [a]_n + [k]_n \cdot [[n,m]]_n =\\
    &= [a+k\cdot 0]_n = [a]_n
\end{align*}
Lo stesso procedimento vale per $m$: $[c']_m = [b]_m$. Dunque $c'\in S$.
Avendo mostrato che $S\subseteq[c]_{[n,m]}$ e $[c]_{[n,m]}\subseteq S$,
allora $S = [c]_{[n,m]}\subseteq\mathbb{Z}$.
\cvd
\end{tcolorbox}

\section{Invertibilità}
\section{La funzione $\Phi$ di Eulero}
\section{\underline{Teorema di Fermat-Eulero}}
\section{\underline{RSA}}


\section{Metodi di calcolo: orbita di una classe}
Le proprietà dell'aritmetica modulare consentono di effettuare calcoli (in modulo)
a mano nonostante ci si dovesse imbattere in numeri estremamente grandi.
Anche se il calcolo manuale non riscuote molto successo al giorno d'oggi,
esistono ancora alcuni sistemi, come le calcolatrici, che impiegano
tecniche di aritmetica modulare per semplificare i calcoli e trattare
numeri altrimenti enormi rispetto alla capacità di rappresentazione digitale
del dispositivo.
Tecniche di calcolo simili si fondano su alcuni principi esaminati in
questa parte:
\begin{itemize}
\item equivalenza tra classi di resto;
\item somma e prodotto tra classi di resto;
\item teorema di Fermat-Eulero, insieme a generalizzazioni e corollari;
\item orbita.
\end{itemize}
Supponiamo di voler calcolare il resto della divisione intera tra
$4^{2465}$ e $3$. Calcolare esplicitamente il dividendo non sarebbe
molto efficiente, ma, conoscendo ora l'aritmetica modulare, sappiamo
che il problema è interpretabile come segue:
\[ \text{si trovi il minimo } r\in\mathbb{Z},r\geq0: [r]_3 = [4^{2465}]_3 \]
Dunque applichiamo le proprietà algebriche: $[4^{2465}]_3 = [4]^{2465}_3 = [1]_3^{2465} = [1^{2465}]_3 = [1]_3 \Rightarrow r = 1$.
Supponiamo ora di voler identificare la cifra delle unità di $7^{222}$.
Con un po' di ragionamento si imposta il problema come segue\footnote{In un naturale espresso in
base decimale, la cifra delle unità corrisponde al resto della divisione tra il numero stesso e 10.}:
\[ d\in\{0,...,9\}: 7^{222}\equiv d\Mod{10} \]
Notiamo che 7 e 10 sono coprimi, quindi per il teorema di Fermat-Eulero
$7^{\Phi(10)}\equiv 1 \Mod{10}$, dove $\Phi(10)=\Phi(2\cdot5)=4$. Ma
allora $7^{222}\equiv 7^{4\cdot55+2} \equiv (7^4)^{55}\cdot7^2\equiv 1^{55}\cdot 49 \equiv 49\equiv 9 \Mod{10} \Rightarrow d=9$.

Illustriamo il concetto di orbita di una classe di resto con la seguente tabella,
prendendo l'esempio di $[8]_{77}$:
\begin{center}
\begin{tabular}{c|l}
    \head{$k$} & \head{Rappresentante di $[8^k]_{77}$}\\
    \hline
    1        & $8$\\
    2        & $64$\\
    3        & $50$\\
    4        & $15$\\
    5        & $43$\\
    6        & $36$\\
    7        & $57$\\
    8        & $71$\\
    9        & $29$\\
    10       & $1$\\
    11       & $8$\\
    12       & $64$\\
    \dots    & \dots
\end{tabular}
\end{center}
Notiamo che all'esponente 10 abbiamo ottenuto il rappresentante
1. Per i principi dell'aritmetica modulare ciò accade sempre. Ma
quello che è ancor più sorprendente è che questa successione è periodica
(provate a proseguire con i calcoli con $k>10$): questa è l'orbita
della classe $[8]_{77}$ elevata a potenze positive. La formulazione
rigorosa di questa proprietà si collega al teorema di Carmichael.

La tecnica dell'orbita permette, in alcuni casi, di velocizzare i calcoli in
modulo in due modi: trovare rappresentanti canonici ridotti e agevoli
da maneggiare o raggiungere la classe 1 senza ripiegare sul teorema
di Fermat-Eulero. Questa tecnica può tornare utile in alcuni
contesti (si vedano gli esercizi) ma non è sempre conveniente. Per esempio,
l'orbita potrebbe essere più lunga di quello che si potrebbe pensare.