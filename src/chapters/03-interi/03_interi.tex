\chptr{$\mathbb{Z}$ e la Divisione}

\section{Qualche parola sui numeri interi}
L'insieme dei numeri interi si denota col simbolo $\mathbb{Z}$
(dal tedesco \textit{Zahl}). Non verrà qui trattata nel dettaglio
la sua struttura, al contrario dei naturali, ma ci limitiamo a
dire che gli interi e le loro operazioni sono definibili per
estensione dai naturali.

\section{\underline{La divisione euclidea}}
\begin{tcolorbox}[enhanced, breakable, title={Teorema della divisione euclidea}]
\[
    n,m \in \mathbb{Z}: m\not=0 \quad \Longrightarrow \quad \exists! q,r\in \mathbb{Z}:
    \begin{cases}
        n = qm + r\\
        0 \leq r < |m|
    \end{cases}
\]
\emph{\textbf{Dimostrazione:}} \underbar{\emph{Esistenza:}} procediamo
per casi. Si consideri dapprima $n,m \in \mathbb{N}$ ($n \geq 0, m > 0$)
e si proceda per induzione, seconda forma, su $n\in\mathbb{N}$.

Se $n = 0$ (base dell'induzione) è sufficiente prendere $q = 0 = r$. Il
sistema della tesi è infatti soddisfatto.

Si supponga ora che $n > 0$ e che la tesi sia vera $\forall k < n$.
Se $n < m$ basta prendere $q = 0$ e $r = n$ (anche qui le condizioni
del sistema sono soddisfatte). Altrimenti si ponga $k = n - m$, dato
che $m > 0$ e pertanto si ha $0 \leq k < n$,
quindi per ipotesi induttiva $\exists q,r \in \mathbb{Z}$ tali che:
\[ 
    \begin{cases}
        k = qm + r\\
        0 \leq r < m = |m|
    \end{cases}
\]
Ma allora $n = k + m = qm + r + m = (q+1)m + r$ e il passo induttivo
è verificato.

Si consideri ora il caso $n < 0$ e $m > 0$. Allora $-n > 0$ e quindi
per il caso precedente esistono $q,r \in \mathbb{Z}$ tali che
$-n = qm + r$ e $0 \leq r < m = |m|$. Per ricondurci alla forma della
tesi, si pone $n = (-q)m - r$. Per sistemare $r$, nel caso $r = 0$ si
soddisfano le condizioni per raggiungere la tesi, se $0 < r < m$ allora
$0 < m - r < m = |m|$ ($m - r$ è pertanto il resto cercato) e:
\[ n = (-q)m - r = (-q)m - m + m - r = (-q - 1)m + (m - r) \]
Infine, se $m < 0$ allora $-m > 0$, quindi per i due casi precedenti
$\exists q,r \in \mathbb{Z}$ tali che:
\[ n = q(-m) + r = (-q)m + r \qquad \text{con } 0 \leq r < -m = |m| \]
\underbar{\emph{Unicità:}} supponiamo che
$qm + r = n = q'm + r'$ e che, a meno di scambiare $r$ e $r'$, $0 \leq r,r' < |m|$.
Supponiamo che $r' \geq r$, allora:
\[ qm + r = q'm + r' \Longleftrightarrow (q - q')m = r'-r \]
\[ \therefore \]
\[ |q-q'||m| = |r'-r| = r'-r < |m| \Longleftrightarrow 0 \leq |q-q'| < 1 \]
Ma dal momento che $q,q' \in \mathbb{Z}$:
\[
    |q-q'| = 0 \Longleftrightarrow q = q' \quad \Longrightarrow \quad qm + r = q'm + r' \Longrightarrow r = r'
\]

\cvd
\end{tcolorbox}

La dimostrazione del teorema costituisce il procedimento alla base
dell'algoritmo ricorsivo per il calcolo del quoziente e del resto
della divisione intera.
\begin{tcolorbox}[enhanced, breakable, colback=red!30, colframe=red!30!black, title=Algoritmo di divisione euclidea]

\end{tcolorbox}

\section{\underline{Rappresentazione dei naturali in base} $b$}
\begin{tcolorbox}[colback=yellow!30, colframe=yellow!30!black, title=Rappresentabilità dei naturali in base arbitraria]
Sia $b\in\mathbb{N}$. Diremo che $n\in\mathbb{N}$ è rappresentabile in base $b$ se
esistono $\varepsilon_0,...,\varepsilon_k\in I_b=\{0,...,b-1\}$ tali che:
\[ n=\varepsilon_0b^0+...+\varepsilon_k b^k = \sum_{i=0}^{k}\varepsilon_ib^i \]
\end{tcolorbox}

\begin{osservaz}
È importante notare che nessun numero è rappresentabile in base 0, perché
$I_0 = \varnothing$ e dunque non esistono simboli che possono rappresentare
i numeri. L'unico naturale rappresentabile in base 1 è 0.
\end{osservaz}

La rappresentabilità è esprimibile in maniera alternativa, come si vede dall'enunciato
del seguente teorema:

\begin{tcolorbox}[enhanced, breakable, title={Teorema di rappresentazione dei naturali in base arbitraria}]
Sia $b \in \mathbb{N}:b \geq 2$. Allora ogni
$n \in \mathbb{N}$ è rappresentabile univocamente nella base $b$.
Ossia esiste una successione $\{\varepsilon_i\}_{i \in \mathbb{N}}$:
\begin{enumerate}
    \item definitivamente nulla ($\exists i_0 \in \mathbb{N}: \varepsilon_i = 0 \quad \forall i > i_0$)
    \item $\varepsilon_i \in I_b \quad \forall i \in \mathbb{N}$ ($0 \leq \varepsilon_i < b$, di tipo resto)
    \item $n = \sum_{i = 0}^{\infty} \varepsilon_i b^i$ (per la (1) tale serie è finita)
\end{enumerate}
e se $\{\varepsilon'_i\}_{i \in \mathbb{N}}$ è un'altra
successione che soddisfa gli stessi punti, allora $\varepsilon_i =
\varepsilon'_i \quad \forall i \in \mathbb{N}$.
$\\\\$
\emph{\textbf{Dimostrazione:}} Sia fissato $b\in\mathbb{N}:b\geq2$.

\underbar{\emph{Esistenza:}} Si procede per induzione, seconda forma, su $n \in \mathbb{N}$.

Se $n = 0$ basta scegliere una successione tale che $\varepsilon_i = 0 \quad \forall i \in \mathbb{N}$.

Assumiamo ora $n > 0$ e si supponga che la tesi sia vera $\forall k < n$.
Si consideri la divisione euclidea tra $n,b$: siano $q,r$ tali che $n = qb + r$
con $0 \leq r < b$. Essendo inoltre $b \geq 2$ vale: $0 \leq q < qb \leq qb + r = n$
e quindi, per ipotesi di induzione, eiste una successione definitivamente
nulla $\{\delta_i\}$ costituita di interi tali che
$0 \leq \delta_i < b \quad \forall i$
e tale che $q = \sum_{i=0}^{\infty}\delta_i b^i$. Segue che:

\[ n = qb + r = \left(\sum_{i=0}^{\infty}\delta_i b^i\right)b + r = \sum_{i=0}^{\infty}\delta_i b^{i+1} + r = \sum_{j=1}^{\infty} \delta_{j-1}b^j + \varepsilon_0 = \sum_{i=0}^{\infty}\varepsilon_i b^i \]

(Si noti che si è posto $\varepsilon_0 = \varepsilon_0 b^0 = r$ e
$\varepsilon_i = \delta_{j-1} \quad \forall i,j > 0$).
La successione $\{\varepsilon_i\}$ è definitivamente nulla, essendo
tale $\{\delta_i\}$ e inoltre $0 \leq \varepsilon_i = \delta_{j-1} < b
\quad \forall i,j > 0$ e $0 \leq \varepsilon_0 = r < b$.
$\newline$
\underbar{\emph{Unicità:}} Si procede per induzione su $n \in \mathbb{N}$.

Se $n = 0 = \sum_{i} \varepsilon_i b^i$ allora ogni addendo della somma,
essendo non negativo ($b \geq 2, \varepsilon_i \in I_b \quad \forall i$), necessariamente deve essere nullo, pertando si
ha $\varepsilon_i = 0 \quad \forall i$.

Si assuma ora $n > 0$ e si suppanga che, $\forall k < n$, l'espressione
in base $b$ sia unica. Sia $n$ tale che:
\[
    n = \sum_{i=0}^{\infty}\varepsilon_i b^i = \sum_{i=0}^{\infty}\varepsilon'_i b^i \Longrightarrow
    n = \left(\sum_{i=1}^{\infty}\varepsilon_i b^{i-1}\right)b + \varepsilon_0 = \left(\sum_{i=1}^{\infty}\varepsilon'_i b^{i-1}\right)b + \varepsilon'_0
\]
Per l'unicità della divisione euclidea, si ha che $\varepsilon_0 = \varepsilon'_0$
e $\sum_{i=1}^{\infty}\varepsilon_i b^{i-1} = q = \sum_{i=1}^{\infty}\varepsilon'_i b^{i-1}$.
Vale $q < n$ (si veda l'esistenza) e pertanto, per ipotesi di induzione,
$\varepsilon_i = \varepsilon'_i \quad \forall i \geq 1$.
\cvd
\end{tcolorbox}
In informatica si incontrano spesso numeri rappresentati in basi
diverse da quella decimale, in modo particolare binaria, ottale
ed esadecimale. Tutto questo è possibile perché il teorema precedente
lo garantisce, sia nella capacità di rappresentare i numeri che
nell'unicità di tali rappresentazioni.

Come per la divisione euclidea, la dimostrazione di questo teorema
mostra un procedimento ricorsivo per il calcolo della stringa che
rappresenta il numero naturale preso in esame data una certa base.

\begin{tcolorbox}[enhanced, breakable, colback=red!30, colframe=red!30!black, title = {Algoritmo di conversione tra rappresentabilità dei naturali in basi diverse}]

\end{tcolorbox}


\section{Proprietà della divisibilità in $\mathbb{Z}$}
Finora abbiamo incontrato la nozione di \textit{divisione}
attraverso quella euclidea, che tuttavia può dare origine
al cosiddetto resto $r$. Un caso interessante di questo
tipo di divisione è proprio quello in cui risulta $r = 0$:
intuitivamente, qualora il divisore ``ci sta esattamente un certo numero
(intero) di volte'' nel dividendo. Questo è il concetto di
\textit{divisibilità} in $\mathbb{Z}$, di cui introduciamo ora la definizione
formale.

\begin{tcolorbox}[colback=yellow!30, colframe=yellow!30!black, title=Divisibilità]
Dati $n,m \in \mathbb{Z}$ si dice che $n$ è un divisore di
$m$ (o che $m$ è un multiplo di $n$) se esiste un $k \in \mathbb{Z}$
tale che $m = kn$:
\[ n|m \quad \Longleftrightarrow \quad \exists k \in \mathbb{Z}: m = kn\]
\end{tcolorbox}

\begin{osservaz}
Dalla definizione valgono le seguenti:
\begin{enumerate}
    \item $n|0 \quad \forall n \in \mathbb{Z}$
    \item $n \not = 0 \Longrightarrow 0 \not | n$
    \item $0|0$
    \item $\pm 1 | n$, $\pm n|n \qquad \forall n \in \mathbb{Z}$
\end{enumerate}
\end{osservaz}

\begin{tcolorbox}[title={Proprietà della divisibilità}]
Dati $n,m\in\mathbb{Z}$ valgono
\begin{enumerate}
    \item $n|m$, $m|q \Longrightarrow n|q$
    \item $n|m$, $m|n \Longrightarrow n = \pm m$
\end{enumerate}
\textit{\textbf{Dimostrazione:}}
\begin{enumerate}
    \item $n|m \Rightarrow m = kn$, $m|q \Rightarrow q = hm = (hk)n \Longrightarrow n|q$
    \item $n|m \Rightarrow m = kn$, $m|n \Rightarrow n = hm \Longrightarrow m = khm \Longleftrightarrow m(1-kh) = 0 \quad \therefore \quad$ $m = 0, n = 0 \because n = hm$ oppure
    $1 - kh = 0 \Longrightarrow h = k = \pm 1 \therefore n = \pm m$
\end{enumerate}
\cvd
\end{tcolorbox}

\begin{tcolorbox}[colback=violet!30, colframe=violet!30!black, title={Lemma utile}]
Siano $c,n,m\in\mathbb{Z}:c|n,c|m$. Allora $c$ divide ogni combinazione lineare
in $\mathbb{Z}$ tra $n$ e $m$:
\[ c|xn+ym \quad \forall x,y\in\mathbb{Z} \]
\textit{\textbf{Dimostrazione:}}

\begin{align*}
    c|n,c|m \Rightarrow&\\
    \Rightarrow &\exists h,k\in\mathbb{Z}:n=ch,m=ck \Rightarrow\\
    \Rightarrow &xn+ym=xch+yck=c(xh+yk) \Rightarrow\\
    \Rightarrow &c|xn+ym
\end{align*}

\cvd
\end{tcolorbox}


\section{\underline{Definizione, esistenza e unicità del MCD}}
\section{Algoritmo di Euclide per il calcolo del MCD}
\section{Numeri coprimi e primi}
\section{\underline{Definizione, esistenza e unicità del mcm}}


\section{\underline{Teorema fondamentale dell'aritmetica}}
\begin{tcolorbox}[enhanced, breakable, title={Teorema Fondamentale dell'Aritmetica}]
Ogni naturale $n \geq 2$ può essere fattorizzato in numeri primi (positivi),
ovvero può essere scritto come prodotto di numeri primi eventualmente ripetuti:
\[n = p_1p_2 \dots p_a  \quad (a \geq 1 \text{ e } p_i \text{ primo}) \]
Inoltre questa scrittura è unica a meno di riordinamento, ovvero: se
$n = q_1 \dots q_b$, dove $q_j$ con $1\leq j \leq b$ sono primi, allora esiste
una bigezione $\varphi: \{1,...,a\} \to \{1,...,b\}$ tale che
$q_j = p_{\varphi(j)}$.
$\\\\$
\emph{\textbf{Dimostrazione:}} \underbar{\emph{Esistenza:}} Si procede per induzione di seconda
forma su $n \geq 2$, considerando la proposizione $P(n) = $ "il numero naturale $n$ si può scrivere
$n = p_1...p_a$ per qualche $p_1,...,p_a$ primi eventualmente ripetuti".

$n = 2$ (base): $n = 2$ è un numero primo, quindi $a = 1$, $p_1 = n = 2$, quindi
$n = p_1 = 2$.

$n > 2, \forall 2 \leq k < n \Rightarrow n$ (passo induttivo): sia $n > 2$.
Assumiamo di saper scrivere ciascun $2\leq k < n$ come
prodotto di numeri primi eventualmente ripetuti (ipotesi induttiva). Dobbiamo
provare che anche $n$ ammette una tale fattorizzazione. Se $n$ è primo, allora
$a = 1$ e $p_1 = n$, quindi $n = p_1$. Supponiamo che $n$ non sia primo, allora
$n = d_1d_2$ con $2 \leq d_1,d_2 < n$, quindi per ipotesi induttiva $d_1 = p_1...p_a$
e $d_2 = q_1...q_b$ ($a,b\geq1$ con $p_1,...,p_a,q_1,...,q_b$ numeri primi eventualmente
ripetuti). Ma allora $n = d_1d_2 = p_1...p_aq_1...q_b$.

\underbar{\emph{Unicità:}} Supponiamo che esista $n \geq 2$ con due fattorizzazioni:
$n = p_1...p_a = q_1...q_b$ per qualche $p_1,...,p_a$, $q_1,...,q_b$ primi
eventualmente ripetuti. Dobbiamo provare che $a=b$ e $p_i = q_i \quad \forall
i \in \{1,...,a\}$ a meno di riordinamento. Osserviamo che $a \leq b$. Si
procede per induzione (prima forma) su $a \geq 1$ considerando $P(a) =$ "per
ogni $b \geq a$, per ogni $p_1,...,p_a,q_1,...,q_b$ primi tali che $p_1...p_a = q_1...q_b$;
allora $a=b$ e, a meno di riordinamento, $p_i = q_i \quad \forall i \in \{1,...,a\}$".

$a=1$ (base) Sia $b \geq 1$ e siano $p_1,q_1,...,q_b$ numeri primi tali che
$p_1 = q_1...q_b$. Dobbiamo provare che $b=1$ e $p_1 = q_1$. Supponiamo per
assurdo che $b \geq 2$. Vale: $p_1 = q_1(q_2...q_b)$, cioè $q_1|p_1$. Ma dal
momento che essi sono numeri primi, deve essere $p_1 = q_1$,
ma qindi $1 = q_2...q_b$, che è assurdo.

$a \geq 1, a \Rightarrow a+1$ (passo induttivo) Supponiamo che vengano dati $p_1,...,p_a,
q_1,...,q_b$ numeri primi con $b \geq a$ e $p_1...p_a = q_1...q_b$. Allora
assumiamo che $a=b$ e, a meno di riordinamento, $p_i = q_i \quad \forall i \in \{1,...,a\}$
(ipotesi induttiva). Dobbiamo dimostrare che: $P(a+1)= $"sia $b \geq a+1$ e siano
$p_1,...,p_{a+1},q_1,...,q_b$ primi tali che $p_1...p_{a+1} = q_1...q_b$.
Allora $b = a+1$ e $p_i = q_i \quad \forall i \in \{1,...,a+1\}$
a meno di riordinamento". Vale: $p_{a+1}|p_1...p_{a+1} = q_1...q_b$, ma per
l'esercizio 10.1 $\exists i \in \{1,...,b\}: p_{a+1}|q_i$. A meno di riordinare
gli indici, supponiamo qui che $p_{a+1}|q_b$. Ma allora $p_{a+1} = q_b$ (ci si
ritrova nella stessa situazione della base). Segue che:
\[ p_1...p_ap_{a+1} = q_1...q_{b-1}q_b \quad \Rightarrow \quad  p_1...p_a = q_1...q_{b-1} \]
Ma $a \leq b-1 \Leftrightarrow a+1 \leq b$: per ipotesi induttiva $a = b-1$ e,
a meno di riordinamento, $p_i = q_i \quad \forall i \in \{1,...,a\}$. Pertanto
$a+1 = b$ e $p_i = q_i \quad \forall i \in \{1,...,a,a+1\}$.

Verificato il passo induttivo, possiamo concludere che, grazie al principio di
induzione di prima forma, la fattorizzazione è unica a meno di riordinamento
$\forall n \geq 2$.
\cvd
\end{tcolorbox}

\subsection*{Perché 1 non è primo?}
Con molta probabilità, alcuni lettori si saranno posti suddetta domanda. Di fatto
1 è divisibile solamente per sé stesso, dunque anche per 1, ma dalla definizione
potrebbe non essere del tutto chiara la ragione per cui i numeri primi
sono maggiori o uguali a 2. Ora che abbiamo illustrato il teorema fondamentale
dell'aritmetica, possiamo rispondere mostrando la ragione matematicamente
formale di questa scelta convenzionale: assumere 1 come numero primo violerebbe
la condizione di unicità della fattorizzazione. Si veda infatti, nella dimostrazione,
la base dell'induzione: non si arriverebbe
all'assurdo se assumessimo $p_1,q_1,...,q_b$ primi incluso 1. Infatti, non ci
sarebbe modo di dare prova che la fattorizzazione in $q_1,...,q_b$ contenga lo
stesso numero di elementi dell'altra.