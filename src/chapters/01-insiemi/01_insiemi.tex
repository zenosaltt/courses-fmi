\chptr{Fondamenti di Insiemistica}

\section{Assiomi}
Ai fini del corso, mostriamo solo alcuni assiomi e concetti primitivi,
appartenenti alla teoria degli insiemi, utili a comprendere ciò che
verrà trattato in seguito.

\begin{tcolorbox}[colback = yellow!30, colframe = yellow!30!black, title = {Elemento, insieme, appartenenza}]
Intenderemo \textit{primitivi} i concetti seguenti:
\begin{itemize}
    \item \textbf{Elemento:} intuitivamente, un oggetto qualsiasi.
    \item \textbf{Insieme:} intuitivamente, una collezione di elementi.
\end{itemize}
Un oggetto si può considerare un insieme se è sempre possibile stabilire
\textit{senza ambiguità} se qualcosa è un suo elemento oppure no. Tale
affermazione esprime la relaizone di \textbf{appartenenza} tra elemento e
insieme:
\[ A \text{ è un insieme } \Longleftrightarrow \forall x,x\in A \text{ o}\footnote{Disgiunzione esclusiva: solo una affermazione può e deve essere vera.}\text{ } x\not\in A \]
\end{tcolorbox}

\begin{osservaz}
Un insieme può a sua volta essere un elemento.
\end{osservaz}

\subsection*{Appartenenza e paradosso di Russell}
Anche se la richiesta espressa dalla condizioine di appartenenza può sembrare
scontata, in realtà può essere fonte di problemi. Consideriamo l'oggetto
\[ R = \{x|x\not\in x\} \]
Per capire se $R$ è un insieme, è necessario passare in rassegna ogni oggetto
$x$ e stabilire se esso appartiene o meno a $R$ (affinché $x\in R$, deve valere
la condizione espressa $x\not\in x$). Quando si dice \textit{ogni oggetto} si
intende anche $R$ stesso. Ecco cosa accade quando si tenta di stabilire se
$R\in R$:
\begin{itemize}
\item Se $R\in R$, allora, per costruzione di $R$, deve essere $R\not\in R$.
\item Se invece $R\not\in R$, vuol dire che $R$ non rispetta la condizione, pertanto $R\in R$.
\end{itemize}
Dunque si giunge a concludere che \[ R\in R \Longleftrightarrow R\not\in R \]
che è un'evidente contraddizione. Quello che è stato appena illustrato è il \textit{paradosso
di Russell}\footnote{Per una versione più divertente del paradosso, invitiamo il
lettore a farsi raccontare la favola-paradosso del barbiere. Non ancora soddisfatti? Provate il paradosso del bibliotecario.}. Grazie agli assiomi precedenti, è allora garantito che $R$ non è
un insieme. Un altro esempio di oggetto non classificabile come insieme,
secondo le definizioni presentate finora, è l'\textit{insieme universo} $\mathcal{U}$,
trattato più avanti.

\begin{tcolorbox}[colback=yellow!30, colframe=yellow!30!black, title={Assioma di Estensionalità}]
Due insiemi sono uguali se possiedono gli stessi elementi.
\[ A = B \Longleftrightarrow (\forall x,x\in A \Leftrightarrow x \in B) \]
\end{tcolorbox}

\begin{tcolorbox}[colback=yellow!30, colframe=yellow!30!black, title=Esistenza dell'insieme vuoto]
Esiste un insieme, chiamato insieme vuoto (indicato col simbolo $\varnothing$), che non possiede elementi.
\[\exists \varnothing:\forall x, x\not\in\varnothing\]
Tale insieme è unico (si dimostra attraverso l'Estensionalità).
\end{tcolorbox}

\begin{tcolorbox}[colback=yellow!30, colframe=yellow!30!black, title={Sottoinsiemi}]
Siano $X,Y$ due insiemi. Si dice che $X$ è sottoinsieme di $Y$
(equivalentemente si può dire \emph{è contenuto}), utilizzando il simbolo $\subseteq$,
se ogni elemento di $X$ è elemento di $Y$.
\[ X \subseteq Y \Longleftrightarrow \forall x(x\in X \Rightarrow x\in Y) \]
\end{tcolorbox}

\begin{osservaz}
Esistono molti simboli per indicare i
sottoinsiemi,: $\subset,\subseteq,\subseteqq$.
Si presti bene attenzione a $\subset$: la letteratura
accademica è piuttosto contraddittoria sul significato di tale
simbolo e non è ben chiaro se esso possa essere utilizzato per
indicare unicamente i sottoinsiemi \textbf{propri}, ovvero quelli
che non possono coincidere col sovrainsieme. Per evitare
ambiguità si utilizzano equivalentemente i simboli $\subsetneq,\subsetneqq$, poco usati
qui.
\end{osservaz}

\section{Insiemi e proprietà: l'assioma di separazione}
Un modo di definire alcuni insiemi è quello di impiegare una \textit{proprietà}
che ne caratterizzi gli elementi. Una proprietà non è altro che una proposizione,
cioè un'affermazione vera e propria, come
\[ P(x) := \textit{ x è fan dei Caesars} \]
Ovviamente a noi interesseranno affermazioni relative a oggetti matematici.
In generale $P(x)$ può essere vera o falsa a seconda di $x$, ma asserzioni come
\[ \forall x \quad x\in\varnothing\Longrightarrow P(x) \]
sono sempre vere. Sembra assurdo, ma approfittiamo di questo esempio per
spiegare che il simbolo $\Longrightarrow$ (\textit{se [ipotesi] allora [tesi]}) esprime, in matematica, \textit{l'implicazione logica materiale}.
In altre parole, l'intera implicazione è vera se l'ipotesi è falsa oppure se
è vera la tesi. \textit{Ex falso quodlibet}, dal falso segue ogni cosa. Nel
nostro caso, la tesi $x\in\varnothing$ è sempre falsa, quindi possiamo
supporre che tutti gli elementi dell'insieme vuoto sono fan dei Caesars.


Se $P$ è una proposizione esprimibile
in termini insiemistici, cioè attraverso gli opportuni simboli e connettori
logici, allora scrivendo
\[\{x|P(x)\}\]
si intende la collezione di tutti gli $x$ che soddisfano $P$. Il paradosso
di Russel in realtà è prova che, in generale, un tale oggetto non è
sempre un insieme. Perciò occorre l'assioma della separazione, che tuttavia
non permette di costruire tutti gli insiemi della teoria degli insiemi, i
quali devono essere definiti con altri assiomi
(tali insiemi esulano da questo corso).

\begin{tcolorbox}[colback=yellow!30, colframe=yellow!30!black, title={Assioma della separazione}]
Se $X$ è un insieme e $P$ una proprietà esprimibile come proposizione
scritta in termini del linguaggio della teoria degli insiemi, allora
la collezione
\[ \{x|x\in X, P(x)\} \]
è un insieme. Alternativamente, vale anche la notazione $\{x\in X|P(x)\}$.
\end{tcolorbox}

Possiamo ora spiegare perché non esiste $\mathcal{U}$, l'insieme di tutti
gli insiemi. Supponiamo invece che $\mathcal{U}$ esista; allora possiamo
scrivere
\[ \{x|x\not\in x\} = \{x\in\mathcal{U}|x\not\in x\} \]
Questo perché $x$, nel membro di sinistra, è anche un insieme; dal momento
che $\mathcal{U}$ è un insieme di insiemi, allora $x\in\mathcal{U}$.
Ma per il paradosso di Russell, il primo membro non può essere un insieme;
quello di destra, per l'assioma di separazione, invece lo è, perché abbiamo
supposto l'esistenza di $\mathcal{U}$\footnote{Il paradosso si supera
introducendo la nozione di \textit{classe}, che tuttavia non vedremo qui.}.

\section{Operazioni tra insiemi}
Se $X$ e $Y$ sono insiemi, si possono ricavare altri insiemi, per mezzo
dell'assioma della separazione, attraverso le seguenti operazioni:
\begin{enumerate}
\item \textbf{Intersezione:} $X\cap Y=\{x|x\in X, x\in Y\}$
\item \textbf{Differenza:} $X\setminus Y=\{x|x\in X,x\not\in Y\}$
\item \textbf{Unione:} $X\cup Y=\{x|x\in X \vee x\in Y\}$
\item \textbf{Prodotto:} $X\times Y=\{(x,y)|x\in X, y\in Y\}$
\item \textbf{Potenza:} $2^X=\{x|x\subseteq X\}$
\end{enumerate}
Se $I$ è un insieme e per ogni $i\in I$ è dato un insieme $X_i$
(si dice che gli $X_i$ sono indicizzati su $I$, cioè $I$ serve
solamente a \emph{numerare} o etichettare gli $X_i$), ridefiniamo
unione e intersezione al caso generale di insiemi multipli:
\begin{enumerate}
\item \textbf{Intersezione:} $\bigcap_{i\in I}X_i=\{x|\forall i \quad x\in X_i\}$
\item \textbf{Unione:} $\bigcup_{i\in I}X_i=\{x|\exists i:x\in X_i\}$
\end{enumerate}
A parole, l'intersezione include gli $x$ appartenenti \textit{ad ogni}
$X_i$, mentre l'unione include gli $x$ contenuti in \textit{almeno un}
$X_i$.


\section{Relazioni e funzioni}
\begin{tcolorbox}[colback=yellow!30, colframe=yellow!30!black, title={Relazione}]
Siano $X,Y$ insiemi. Si dice relazione ($\mathcal{R}$) tra $X$ e $Y$ un sottoinsieme
del prodotto tra $X$ e $Y$.
\[\mathcal{R}\subseteq X\times Y\]
Si scriverà anche $x\mathcal{R}y$ per indicare $(x,y)\in\mathcal{R},
\quad x\in X,y\in Y$. Una relazione in $X\times X$, cioè tra $X$ e se stesso,
si chiamerà \textit{relazione binaria} su $X$.
\end{tcolorbox}

Anche nella vita di tutti i giorni, si incontra spesso una classe di relazioni,
dette \textit{relazioni di equivalenza}, che condividono tre proprietà semplici ma per
nulla scontate.

\begin{tcolorbox}[colback=yellow!30, colframe=yellow!30!black, title={Relazione di equivalenza}]
Sia $X$ un insieme e sia $\mathcal{R}$ una relazione binaria su $X$
(cioè $\mathcal{R} \subseteq X \times X$). Diciamo che $\mathcal{R}$ è una
relazione di equivalenza su $X$ se possiede le seguenti proprietà:
\begin{enumerate}
    \item \textbf{Riflessiva:} $\forall x \in X, \quad x \mathcal{R} x$
    \item \textbf{Simmetrica:} $\forall x,y \in X, \quad x \mathcal{R}y \Longrightarrow y \mathcal{R} x$
    \item \textbf{Transitiva:} $\forall x,y,z \in X, \quad x \mathcal{R} y \text{ e } y \mathcal{R} z \Longrightarrow x \mathcal{R} z$
\end{enumerate}
\end{tcolorbox}
La portata di questo genere di relazioni è ampia e copre non solo i linguaggi
matematici, ma anche la logica quotidiana. Ad esempio, qualsiasi cosa è uguale
a sé stessa; se la mia auto ha lo stesso colore della tua, allora la tua auto ha lo stesso
colore della mia; se posso raggiungere Roma in skateboard partendo da Verona,
e posso fare altrettanto da Roma a Foggia, posso viaggiare in skateboard da
Verona a Foggia. In informatica si possono incontrare numerose relazioni di
equivalenza. Generalmente nei corsi di teoria dei linguaggi di programmazione
si tratta \textit{l'equivalenza di tipo}, che è una relazione di equivalenza.
In Java esiste il metodo \texttt{equals} (si occupa del confronto tra oggetti
appartenenti a classi date), la cui definizione deve mandatoriamente
rispettare le proprietà delle relazioni di equivalenza. Tale obbligo spetta
peraltro al programmatore, qualora si decida di effettuare \textit{override}
del metodo.

\begin{tcolorbox}[colback=yellow!30, colframe=yellow!30!black, title={Funzione}]
Una relazione $f\subseteq X\times Y$ si dice \textit{funzione} (o funzione
\textit{totale}) se per ogni $x\in X$ esiste unico $y\in Y$ per cui $xfy$.
\[ f \text{ funzione} \Longleftrightarrow \forall x\in X,\exists!y\in Y:xfy \]
Si scriverà anche $f:X\to Y$ per indicare che $f$ è una funzione (o applicazione)
da $X$ in $Y$ e $y=f(x)$ come sinonimo di $(x,y)\in f$.
\end{tcolorbox}

L'insieme $\{x\in X|\exists y\in Y: y=f(x)\}$ prende il nome di \textit{dominio}
di $f$ ($\text{dom}(f)$), mentre $\{y\in Y|\exists x\in\text{dom}(f):y=f(x)\}$
è l'\textit{immagine} di $f$. In realtà, per la definizione appena data
$\text{dom}(f)=X$, ma ciò è vero per le funzioni \textit{totali}. Esistono
anche funzioni \textit{parziali}, nelle quali per ogni $x\in X$ esiste \textit{al più}
un $y\in Y$ col quale $x$ è in relazione. Per noi saranno importanti le funzioni
totali.

Denotiamo con $Y^X$ l'insieme di tutte le funzioni totali da $X$ a $Y$, cioè
\[ Y^X = \{f\in2^{X\times Y}| \forall x\in X, \exists!y\in Y: xfy\} \]
Se $X$ è un insieme, $\text{id}_X = \{(x_1,x_2\in X\times X| x_1=x_2)\}$ è
una funzione, chiamata \textit{identità} di $X$ e vale \[ \text{id}_X(x)=x \quad \forall x\in X\]
Se $f:X\to Y$ è una funzione e $A\subseteq X$, allora $f\cap(A\times Y)$
è ancora una funzione, chiamata \textit{restrizione di $f$ ad $A$} e indicata
con $f|_A$. In altre parole, essa rappresenta una parte della funzione $f$,
ristretta ad un sottoinsieme del suo dominio.

\begin{tcolorbox}[colback=yellow!30, colframe=yellow!30!black, title={Composizione}]
\end{tcolorbox}
\begin{tcolorbox}[colback=yellow!30, colframe=yellow!30!black, title={Iniettività, surgettività, bigettività}]
\end{tcolorbox}




\section{Equipotenza di insiemi}
