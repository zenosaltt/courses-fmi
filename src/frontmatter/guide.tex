\chapter*{Note e Guida al Testo}
Gli argomenti trattati appartengono alla branca della Matematica Discreta,
quindi quella matematica che si occupa non del \textit{continuo} ma di
ciò che è \textit{numerabile} o talvolta \textit{finito}. In parole
semplici, di ciò che si può contare con le dita di una mano. Il corso
è diviso in due parti:
\begin{itemize}
\item \textbf{Aritmetica Modulare:} dalla teoria degli insiemi,
si passerà ai numeri naturali e interi, all'operazione di divisione,
il concetto di congruenza in modulo e infine all'applicazione
del metodo di crittografia RSA.

\item \textbf{Introduzione alla Teoria dei Grafi:} verrà data un'occhiata
ad alcuni grafi, se ne daranno definizioni formali (forse meno familiari
rispetto alla loro controparte grafica) e proprietà rilevanti.
\end{itemize}
In più, vengono aggiunte una terza parte di esercitazione, che riflette
lo schema di lezioni dell'A.A. 2022/2023, ed una quarta che include
alcune appendici.

Questi appunti rispettano ancora fedelmente il loro iniziale mandato:
raccogliere tutti i teoremi utili all'esame, arricchiti con le spiegazioni
e le indicazioni del professore. Per questo li troverete evidenziati
all'interno degli appositi box grigi. Col tempo, si sono aggiunte altre
categorie degne di nota altrettanto evidenziate. Ecco le indicazioni di ogni
colore:
\begin{itemize}
\item \textcolor{gray}{Grigio}: teorema o proposizione rilevante.
\item \textcolor{green}{Verde}: corollario.
\item \textcolor{yellow}{Giallo}: definizione.
\item \textcolor{blue}{Blu}: esempio esplicativo o esercizio con soluzione.
\item \textcolor{red}{Rosso}: nozione utile.
\item \textcolor{violet}{Viola}: il lemma utile.
\end{itemize}
I teoremi richiesti al tempo di stesura sono sottolineati, sia
nei paragrafi che nell'indice, per una veloce consultazione.
Affidarsi all'indice, non solo all'indicazione visiva dei box
grigi.

Si fa qui intenso utilizzo di simboli logico-matematici, sia per
contrarre i predicati e le formule che per produrre fedelmente la
scrittura ideale da adottare all'esame e al di fuori di esso. Si
è posta attenzione, per quanto possibile, ad affiancare a tali
simboli una spiegazione verbale concisa. Sicuramente il testo non
è del tutto privo di abusi di notazione o simbologia non-standard.
Per questi ed altri motivi, \textit{gli apppunti non sostituiscono
interamente le lezioni frontali}. Eventuali spiegazioni più chiare
e precise sono sicuramente fornite, nella stragrande maggioranza dei
casi, dal professore. Questa dispensa mira solo a riassumere i contenuti
delle lezioni.