\chapter*{Prefazione}

La presente dispensa, tavola di contenuti utili o meno, raccolta di appunti
o qualsivoglia supporto allo studio, nasce come disordinata e incoerente collezione di note
trascritte nel corso delle lezioni, sessioni di studio ed esercitazioni relative al
corso \textit{Fondamenti Matematici per l'Informatica}, durante l'anno accademico
2022/2023 presso l'Università degli Studi di Trento. Evolutasi a raccolta dei teoremi
fondamentali richiesti all'esame, questa dispensa si mostra ora come testo organico e
dignitosamente ordinato. Non si può negare il fatto che fossero già presenti materiali
tramandati da generazioni di studenti durante la stesura di questi appunti, ma sentivamo
forte l'esigenza di riunire in un unico luogo non solo i teoremi, ma anche definizioni,
esempi esplicativi di esercizi d'esame e ulteriori spiegazioni teoriche, talvolta solo
citate oralmente a lezione o difficilmente reperibili.

Ma qui non si discute di mero esercizio formale, di sfoggio di conoscenza o della
presunta padronanza di strumenti tipografici come \LaTeX. Questa dispensa si propone
innanzitutto come risorsa (si spera) utile agli studenti presenti e futuri, del tutto
gratuita. La scienza è umana ed è all'umanità che essa viene messa a disposizione.
In secondo luogo, non è garantita l'assenza di errori, siano essi grammaticali, lessicali,
sintattici, grafici, concettuali, di interpretazione, di conto, distrazioni e così via.
Infine questo testo non pretende, e non potrebbe minimamente farlo, di essere un libro
di testo; perché le nozioni qui mostrate e trascritte non sono altro che il frutto del
lavoro di studenti, non professori o esperti in merito. Nonostante ciò, nell'ordinare
tutti questi appunti è stata prestata la maggior attenzione nel riportare fedelmente il
significato e il pensiero matematico di questa materia straordinaria.

Ultimo, ma forse il più caro tra i motivi che ci hanno spinto a creare queste note, è
la speranza che questo corso appassioni gli studenti di informatica, sia per quanto riguarda
la matematica che per l'informatica stessa, perché l'informatica è progenie della matematica
come la fisica lo è per la filosofia. Certo, molti degli argomenti mostrati sono solo un
assaggio di quel vasto mondo della matematica discreta e per molti sembrerà solo un
conglomerato di concetti astratti, oltre che apparentemente inutili. Ma crediamo nel
seguente postulato

\[ \exists \text{studente}: \text{studente} \in \text{Corso di Informatica}, \text{ studente apprezza la materia} \]

\noindent cioè, crediamo che almeno uno tra i lettori farà tesoro di queste nozioni e che,
come noi, trovi motivo di spingere ancora più in là la propria curiosità, non solo per
l'informatica in sé, ma anche per tutto ciò che ha permesso a questa giovanissima scineza di
progredire, prima fra tutte la matematica.

Non fermatevi solo di fronte alla superficie, perché la totalità del reale non è ancora
stata scoperta a fondo. La tecnica di crittografia RSA, trattata molto brevemente in
queste note, è stata una grande novità e venne sviluppata solamente a partire dagli anni
Settanta del secolo scorso. Finché si ricordava da poco il primo allunaggio, tre persone
pensavano a cosa potevano fare dei semplici numeri interi divisi, moltiplicati, sommati
e sottratti; apparentemente la cosa più banale che si potesse fare. Eppure si fecero
ancora passi avanti in matematica, crittografia e informatica seppur dopo ben 2500 anni
di storia, ovvero dalla trattazione della divisione euclidea, anch'essa citata qui.

Ringraziamo chiunque abbia contribuito, direttamente e indirettamente, a questo progetto:
studenti, con i loro materiali generosamente condivisi e ormai reperibili con gran facilità,
professori, con la chiarezza delle loro lezioni e la disponibilità e l'umorismo, esercitatori,
con i loro segreti sul come passare l'esame. Noi non ci abbiamo messo solo impegno e fatica,
ma curiosità e voglia di saperne ancora di più su questo universo e su noi stessi. Ora tocca
a voi portare avanti la nostra conoscenza, che è la vostra.