\chptr{$\mathbb{N}$, Insiemi Ordinati e Insiemi Finiti}

\section{L'insieme $\mathbb{N}$}
Un insieme numerico fondamentale in matematica discreta—in
particolar modo per questo corso—è quello dei numeri naturali,
indicato con il simbolo $\mathbb{N}$. La sua struttura attuale
è dovuta al matematico italiano Giuseppe Peano, che formalizzò
la natura di tale insieme attraverso i suoi celebri assiomi.
\begin{tcolorbox}[colback=yellow!30, colframe=yellow!30!black, title=Assiomi di Peano]
\begin{enumerate}[(i)]
    \item $\mathbb{N}$ ammette almeno un elemento: 0 (zero). \[ 0 \in \mathbb{N} \]
    \item Esiste una funzione iniettiva avente $\mathbb{N}$ come dominio sia come codominio.
    \[ \exists \text{ succ: } \mathbb{N} \rightarrow \mathbb{N} \text{ iniettiva} \]
    \item Il successivo di ogni naturale non è mai 0.
    \[ \forall n \in \mathbb{N}, \text{succ}(n) \not = 0 \]
    \item \textbf{Assioma di induzione:} Se $A\subseteq\mathbb{N}$ è un sottoinsieme contenente 0 e
    per cui vale tale proprietà: per ogni naturale $n$, se $n$ appartiene
    ad $A$, allora anche $\text{succ}(n)$ appartiene ad $A$; allora $A$
    coincide con $\mathbb{N}$.
    \[ [A \subseteq \mathbb{N}: 0 \in A, \quad \forall n \in \mathbb{N} \quad (n \in A \Rightarrow \text{succ}(n) \in A)]  \Longrightarrow  A = \mathbb{N} \]
\end{enumerate}
\end{tcolorbox}

L'assioma di induzione costituisce il principio fondante di
una delle tecniche di dimostrazione più potenti della matematica:
la dimostrazione per induzione. Questa costituirà infatti la
logica di dimostrazione maggiormente sfruttata in questo corso.


\section{Il principio di induzione di prima forma}
\begin{tcolorbox}[title=Prima forma dell'induzione (A)]
Sia $\{P(n)\}_{n \in \mathbb{N}}$ una famiglia di proposizioni
indicizzata su $\mathbb{N}$. Supposto che:
\begin{enumerate}
    \item $P(0)$ sia vera
    \item $\forall n \in \mathbb{N} \quad P(n) \Longrightarrow P(n+1)$
\end{enumerate}
Allora $P(n)$ è vera $\forall n \in \mathbb{N}$.
$\\\\$
\emph{\textbf{Dimostrazione:}} Sia $A = \{n | P(n)\}$. Per la (1) si ha
che $0 \in A$. Se $n \in A$ allora vale $P(n)$, quindi $P(n+1)$
vale per la (2), ma allora $n+1 \in A$, quindi per l'assioma di
induzione $A = \mathbb{N}$.
\cvd
\end{tcolorbox}

Spesso si desidera limitare la dimostrazione di certe proposizioni
ad un sottoinsieme. Per questo esiste una formulazione alternativa
del principio di induzione, utile a tale scopo.

\begin{tcolorbox}[title=Prima forma dell'induzione (B)]
Sia $h \in \mathbb{N}$ e sia $\{P(n)\}_{n \geq h}$ una famiglia
di proposizioni indicizzata su su $n \geq h$. Supposto che:
\begin{enumerate}
    \item $P(h)$ sia vera
    \item $\forall n \geq h \quad P(n) \Longrightarrow P(n+1)$
\end{enumerate}
Allora $P(n)$ è vera $\forall n \geq h$.
$\\\\$
\textbf{\emph{Dimostrazione:}} Sia $A = \{n \geq h| P(n)\}$.

$n=h$ (base dell'induzione) $n=h \Longrightarrow n \in A \because \text{ip. (1)}$.

$\forall n \geq h, n \Rightarrow n+1$ (passo induttivo) Supposto che qualche $n \in A$,
con $n \geq h$ allora per (2) si ha $P(n) \Rightarrow P(n+1)$, ma allora $n+1 \in A$.
Per l'assioma di induzione, $P(n)$ vera $\forall n \geq h$.
\cvd
\end{tcolorbox}

Come per molti altri teoremi che vedremo più avanti, le dimostrazioni del
principio di induzione forniscono uno schema generale per mostrare la validità
di proposizioni indicizzate sui naturali.




\section{Il teorema di ricorsione}
Con i soli assiomi di Peano non si può fare molta strada: non
definiscono operazioni come somma e prodotto e nemmeno relazioni
d'ordine—poter stabilire se un numero è più grande di un altro—
sugli elementi di $\mathbb{N}$. Per poter procedere, è necessario
citare l'assioma di ricorsione—probabilmente il più complesso del
corso, ma sul quale si fondano operazioni scontate come la somma
di numeri naturali.

\begin{tcolorbox}[title=Teorema di ricorsione]
Sia $X$ un insieme, $h:\mathbb{N}\times X\to X$ una funzione
e $c\in X$. Esiste un'unica funzione $f:\mathbb{N}\to X$ tale
che
\begin{align*}
    f(0) &= c\\
    f(\text{succ}(n)) &= h(n,f(n)) \quad \forall n \in\mathbb{N}
\end{align*}

\textit{\textbf{Dimostrazione:}} non vuoi vedere due pagine di dimostrazione
\end{tcolorbox}


\section{Operazioni fondamentali sui naturali}
Enunciato il teorema di ricorsione, definiamo la somma e il prodotto
in $\mathbb{N}$.
\begin{tcolorbox}[colback=yellow!30, colframe=yellow!30!black, title={Somma in $\mathbb{N}$}]
Dato $n\in\mathbb{N}$ si definisce + la funzione $m\mapsto n+m$ ricorsivamente
nel seguente modo:
\begin{align*}
    n + 0 &= n\\
    n + \text{succ}(m) &= \text{succ}(n + m)
\end{align*}
\end{tcolorbox}

Ponendo $\text{succ}(0) = 1$, grazie al teorema di ricorsione si ottiene
\[ \text{succ}(n) = \text{succ}(n + 0) = n + \text{succ}(0) = n + 1 \]
D'ora in poi verrà utilizzata questa scrittura, più naturale e familiare,
per indicare il successivo del numero $n$.

\begin{tcolorbox}[colback=yellow!30, colframe=yellow!30!black, title={Prodotto in $\mathbb{N}$}]
Dato $n\in\mathbb{N}$ si definisce il prodotto $m\mapsto nm$
\begin{align*}
    n0 &= 0\\
    n(m+1) &= nm + n
\end{align*}
\end{tcolorbox}

\begin{osservaz}
Queste definizioni prevedono solo la
somma e il prodotto a sinistra, ma si può mostrare che vale
la \textit{commutatività}, oltre all'associatività e alla
proprietà distributiva.
\end{osservaz}


\section{Insiemi ordinati}
Definita la somma, possiamo tramite essa formalizzare
l'ordinamento dei numeri naturali.

\begin{tcolorbox}[colback=yellow!30, colframe=yellow!30!black, title={Ordinamento in $\mathbb{N}$}]
Siano $n,m\in\mathbb{N}$. Diremo che $n$ è \textit{minore o guale} a $m$
scrivendo
\[ n\leq m \Longleftrightarrow \exists k\in\mathbb{N}: m = n+k \]
\end{tcolorbox}
L'ordinamento sui naturali $\leq$ può essere visto come sottoinsieme
di $\mathbb{N}\times\mathbb{N}$, cioè \[ \leq \quad=\quad \{(n,m)\in\mathbb{N}\times\mathbb{N}|\exists k\in\mathbb{N}:m=n+k\} \]
e dunque $\leq$ è una relazione sui naturali. Infatti $n\leq m$ ha
lo stesso significato di $(n,m)\in\leq$, seguendo la definizione
insiemistica. Valgono poi le seguenti proprietà
(dimostrabili) $\forall n,m,k\in\mathbb{N}$:
\begin{enumerate}
\item $n\leq n$
\item $n\leq m, m\leq n \Longrightarrow n = m$
\item $n\leq m, m\leq k \Longrightarrow n \leq k$
\item $n\leq m \text{ o } m\leq n$
\end{enumerate}

In generale, un ordine degli oggetti può essere stabilito su
quasi tutti gli insiemi. Infatti il linguaggio insiemistico
descrive le regole per tale costruzione estendendole ad un
generico insieme, come si vede nella prossima definizione.
L'ordinamento dei numeri può sembrare qualcosa di totalmente
naturale, ma in realtà e in generale si tratta solo di una
convenzione. In informatica, ad esempio, esiste una relazione
d'ordine sulle stringhe e i singoli caratteri, che si basa
sull'ordine alfabetico di lettere e parole. Nulla vieta di
stabilire ordinamenti su colori o strumenti musicali. Basta
``inventarsi" una relazione sotto forma di proposizione (possibilmente
formale e non ambigua) come ``$x$ viene prima (o con la stessa
priorità) di $y$". Ma come vedremo è necessario stabilire almeno 3 regole
affinche una relazione d'ordine su un certo insieme sia considerabile tale:
\begin{enumerate}
    \item Ogni elemento deve poter essere confrontabile con sé stesso: $x$ viene prima o con la stessa priorità di sé stessp.
    \item Se un elemento è in relazione con un altro, che è in relazione col primo, allora questi elementi sono uguali: se $x$ viene prima o con uguale priorità di $y$ e $y$ viene prima o con uguale priorità di $x$, allora $x = y$.
    \item Se un elemento è in relazione con un altro, che è in relazione con un terzo, allora il primo è in relazione con il terzo: se $x$ viene prima di $y$ e $y$ viene prima di $z$, allora $x$ viene prima di $z$.
\end{enumerate}

\begin{tcolorbox}[colback=yellow!30, colframe=yellow!30!black, title={Relazione d'ordine parziale e totale}]
Sia $X$ un insieme e $\preceq$ una relazione binaria su
$X$. $\preceq$ si dice un ordinamento parziale, o relazione
d'ordine parziale, se valgono se seguenti proprietà:
\begin{enumerate}
    \item \textbf{Riflessiva:} $\forall x\in X \quad x\preceq x$
    \item \textbf{Antisimmetrica:} $\forall x,y\in X \quad x\preceq y, y\preceq x \Longrightarrow x = y$
    \item \textbf{Transitiva:} $\forall x,y,z\in X \quad x\preceq y, y\preceq z \Longrightarrow x\preceq z$
\end{enumerate}
Un ordinamento parziale si dice totale se in più vale la \textbf{tricotomia}
\[ \forall x,y\in X \quad x\preceq y \text{ o } y\preceq x \]
\end{tcolorbox}
Una coppia $(X,\preceq)$ si dice \textit{insieme parzialmente o totalmente
ordinato}.
I simboli (equivalenti) $<,\prec$ indicano l'\textit{ordinamento stretto}, ossia
$x\prec y, x\not=y$. La proprietà di tricotomia può essere riformulata
in termini di ordinamento stretto scrivendo
\[ \forall x,y\in X \text{ vale una e una sola tra } x\prec y \text{ o } x=y \text{ o } y\prec x \]
Introdotto questo nuovo linguaggio, si può allora affermare che
$(\mathbb{N},\leq)$ è un insieme \textit{totalmente} ordinato.

L'ordinamento totale è forse la relazione d'ordine con cui si ha a che
fare più spesso nella vita: i numeri naturali ne sono una dimostrazione
lampante. Ma quindi quando si incontrano relazioni d'ordine \textit{parziali}?
Ma innanzitutto, cosa caspita significa essere una relazione d'ordine totale? Intuitivamente,
una relazione d'ordine totale definita su uncerto insieme permette di
confrontare \textit{sempre} due elementi \textit{qualsiasi} di
quell'insieme. Cioè, se $\mathcal{R}$ è una generica relazione, ad esempio
``è alfabeticamente minore di'' impiegata su un insieme di parole $W$,
il simbolo $x\mathcal{R}y$ ha sempre significato $\forall x,y\in W$.

Un esempio di relazione d'ordine parziale è invece il seguente:
\[ \mathcal{R} = \text{``È antenato di o identico a''} \]
definito sull'insieme degli esseri umani. È facile verificare che
valgono le proprietà (1), (2) e (3) mostrate precedentemente, ma
non vale la tricotomia: non è vero che \textit{per ogni} essere
umano è possibile stabilire se uno è antenato o identico all'altro.
Due fratelli distinti, infatti, non possono essere confrontati con questa
relazione (si ricorda che si è supposto che l'insieme è quello di tutti
gli esseri umani e che la relazione deve valere per tutte le coppie
possibili di elementi; da qui il significato intuitivo del termine
\textit{totale}).

Se il lettore insaziabile desidera una prova dell'utilità di queste
nozioni in ambito informatico, la struttura dati conosciuta come
\textit{heap} definisce un ordinamento parziale sui suoi elementi.
Maggiori informazioni sono disponibili nel corso di \textit{Algoritmi
e Strutture Dati} del famigerato Alberto Montresus\footnote{Personalità
influente presso l'Università di Trento conosciuta anche come Montresor,
Montresorus e altri pseudonimi.}

\section{Insiemi finiti: il lemma dei cassetti}
Forniamo ora una definizione-notazione usata frequentemente in
questo corso: dato un numero naturale $n\in\mathbb{N}$ denotiamo
$I_n = \{0,1,...,n-1\}$. In particolare $I_0 = \varnothing$.

\begin{tcolorbox}[colback=yellow!30, colframe=yellow!30!black, title={Insieme finito}]
Un insieme $X$ si dice finito se esiste un naturale $n$ tale che
$X$ è equipotente a $I_n$. Un insieme è infinito se non è finito.
\[ X \text{ finito} \Longleftrightarrow \exists n\in\mathbb{N}: X\sim I_n \]
\end{tcolorbox}


\begin{tcolorbox}[enhanced, breakable, title={Lemma dei cassetti}]
Siano $X,Y$ due insiemi aventi rispettivamente $X \sim I_n$ e $Y \sim I_m$
con $n < m$. Allora ogni applicazione $f:Y \to X$ non è iniettiva.
$\\\\$
\textit{\textbf{Dimostrazione:}} Si procede per induzione su $n \in
\mathbb{N}$. Se $n=0$ allora $X=\varnothing$ e $Y\not = \varnothing$,
quindi l'insieme $X^Y$ delle applicazioni $Y\to X$ è vuoto e pertanto
non c'è nulla da dimostrare (dal falso segue ogni cosa).

Supponiamo che la tesi sia vera per $n$ e proviamola per $n+1$. Sia
$X \sim I_{n+1}$ e sia $Y \sim I_m$ con $m > n+1$ e supponiamo per
assurdo che l'applicazione $f:Y\to X$ sia iniettiva. Per definizione
esiste una bigezione $g:I_{n+1} \to X$; poniamo $x_n = g(n)$ e
$X' = X-\{x_n\}$. Chiaramente $X'$ è in bigezione con $I_n$. Si
presentano due casi:
\begin{align*}
    f^{-1}(x_n) = \varnothing      &\quad (\text{i.e. }\forall y \in Y: f(y) \not = x_n)\\
    f^{-1}(x_n) \not = \varnothing &\quad (\text{i.e. }\exists y \in Y: f(y) = x_n)
\end{align*}
Nel primo caso, $f(y)\subseteq X'$ e quindi $f:Y\to X'$ sarebbe
un'applicazione iniettiva da un insieme equipotente a $I_m$ in
un insieme equipotente a $I_n$; dato che $m>n+1>n$ questo è assurdo
per ipotesi di induzione.
Nel secondo caso, sia $y\in Y$ tale che $f(y)=x_n$ e sia $Y'=Y-\{y\}$.
Dato che $f$ è iniettiva, $f(Y') \subseteq X'$ e quindi $f|_{Y'}:
Y'\to X'$ è un'applicazione iniettiva. Dato che $Y'\sim I_{m-1}$,
$X'\sim I_n$ e $m-1>n$, ciò è assurdo per ipotesi di induzione.
\cvd
\end{tcolorbox}

Il lemma dei cassetti prende il nome da un modo intuitivo di interpretarlo:
se possiedo un certo numero di oggetti da disporre in un certo numero di
cassetti, più piccolo di quello degli oggetti, allora sicuramente qualche
cassetto conterrà più di un oggetto.
In informatica questo concetto, seppur banale, assume un'importanza da non
sottovalutare: se, in una funzione, si conoscono perfettamente tutti i possibili output e
tutti i possibili input e se questi sono maggiori del numero di output,
allora, per il lemma dei cassetti, almeno un output sarà condiviso da più
input. Ovvero, due o più input produrranno lo stesso output.

\begin{osservaz}
Qualora dominio e codominio abbiano lo stesso numero di elementi,
è possibile per una tale applicazione essere bigettiva, semplicemente
mostrando la sua iniettività. Di fatto il lemma dei cassetti può essere
riformulato come segue: \emph{se n oggetti vengono distribuiti in n posti
cosicché ogni posto non riceva più di un oggetto, allora ogni posto
conterrà uno e un solo oggetto}. Nel teorema di Fermat-Eulero
si farà riferimento a questa formulazione del
lemma dei cassetti.
\end{osservaz}

Dal lemma dei cassetti seguono corollari e definizioni importanti
relativi alle proprietà degli insiemi finiti.
\begin{tcolorbox}[colback=green!30, colframe=green!30!black, title={Equipotenza tra insiemi finiti}]
Dati $n,m\in\mathbb{N}, n\not=m$ e $X,Y$ insiemi finiti tali che $|X|=|I_n|$ e $|Y|=|I_m|$,
allora $X$ e $Y$ non sono equipotenti.

In particolare vale \[ |X|=|I_n|,|X|=|I_m| \Longrightarrow n=m \]
insieme a \[ X\sim I_n,Y\sim I_m \Longrightarrow (X\sim Y \Leftrightarrow n = m) \]
\end{tcolorbox}
Da quest'utlimo corollario si può definire la cardinalità degli insiemi
finiti.

\begin{tcolorbox}[colback=yellow!30, colframe=yellow!30!black, title={Cardinalità di un insieme finito}]
Sia $X$ un insieme finito. Si dice cardinalità di $X$ l'unico numero
naturale $n$ tale che $|X|=|I_n|$. Tale numero si denota con $|X|$.
\end{tcolorbox}

Riproponiamo la condizione di equipotenza, in particolare per quanto
riguarda gli insiemi finiti ed introducendo il linguaggio della cardinalità.

\begin{tcolorbox}[title={Condizione di equipotenza espressa in termini di cardinalità}]
Dati $X,Y$ insiemi finiti vale
\[ X\sim Y \Longleftrightarrow |X|=|Y| \]
\textit{\textbf{Dimostrazione:}} $|X|=|Y|\Longrightarrow \exists n\in\mathbb{N}:
X\sim I_n, Y\sim I_n\Longrightarrow X\sim Y$ per transitività dell'equipotenza.
Viceversa se $X\sim Y$ allora essi hanno la stessa cardinalità, per il
corollario precedente.
\end{tcolorbox}

\begin{tcolorbox}[title={Cardinalità di sottoinsiemi finiti}]
Sia $X$ un insieme finito e $Y\subseteq X$. Allora anche
$Y$ è finito e $|Y|\leq|X|$. Se $Y\subsetneqq X$, allora
$|Y|<|X|$.
\end{tcolorbox}

\begin{tcolorbox}[colback=green!30, colframe=green!30!black, title={Equipotenza tra insieme finito e sottoinsieme}]
Un insieme finito non è equipotente ad alcun suo sottinsieme proprio,
cioè non esiste alcuna bigezione tra $X$ e un qualsivoglia $X'\subsetneqq X$.
\end{tcolorbox}

\begin{osservaz}
Segue in particolare che $\mathbb{N}$ è
infinito. Per gli assiomi di Peano, si può considerare
l'applicazione $n\mapsto \text{succ}(n)$; sapendo che 0 non
possiede alcun predecessore, $\text{succ}$ è una bigezione
tra $\mathbb{N}$ e il suo sottoinsieme proprio $\mathbb{N}\setminus\{0\}$.
$\mathbb{N}$ dunque non soddisfa il corollario precedente.
\end{osservaz}





\section{\underline{Buon ordinamento}}
\begin{tcolorbox}[colback=yellow!30, colframe=yellow!30!black, title=Minimo]
Sia $X$ un insieme e sia $\leq$ un ordinamento su $X$. Sia $A \subseteq X$, diremo
che $z \in A$ è un minimo di $A$ se per ogni $x \in A$ si ha che $z \leq x$.
\[ z = \min A \Longleftrightarrow z \leq x \quad \forall x \in A \]
\end{tcolorbox}

\begin{osservaz}
Non avendo elementi, l'insieme vuoto non ammette minimo.
\end{osservaz}

\begin{tcolorbox}[colback=yellow!30, colframe=yellow!30!black, title=Insieme ben ordinato]
Un ordinamento totale su un insieme $X$ si dice buon ordinamento se ogni sottoinsieme
non vuoto di $X$ ammette minimo.
\[ (X,\leq) \text{ ben ordinato } \Longleftrightarrow \exists \min A \quad \forall A \subseteq X, A \not = \varnothing \]
\end{tcolorbox}

\begin{tcolorbox}[title={Buon ordinamento di $\mathbb{N}$}]
L'ordinamenti dei numeri naturali è un buon ordinamento.
\[ (\mathbb{N},\leq) \text{ ben ordinato} \]
\emph{\textbf{Dimostrazione:}} Sia $A \subseteq \mathbb{N}$ che non
ammette minimo e mostriamo che $A = \varnothing$.
Sia $B = \mathbb{N}\setminus A$ il complementare di $A$. Si procede per induzione su
$n \in \mathbb{N}$ considerando la proposizione:
\[ P(n) := \left(\{0,1,...,n\} \subseteq B\right) \]
$n=0$ (base dell'induzione) Si osservi che $0 = \min\mathbb{N}$, dunque
$0 \not \in A$, altrimenti sarebbe il suo minimo,
allora $0 \in B$ e di conseguenza $\{0\} \subseteq B$.

$n\geq0, n\Longrightarrow n+1$ (passo induttivo)
Supponiamo che $\{0,...,n\} \subseteq B$ per qualche $n\geq0$ (ipotesi
induttiva). Allora
$0,...,n \not \in A$ per ipotesi induttiva, quindi $n+1 \not \in A$,
altrimenti sarebbe il suo minimo. Quindi $n+1 \in B$ e pertanto
$\{0,...,n,n+1\} \subseteq B$. Allora $B = \mathbb{N}$, quindi
$A = \varnothing$.
\cvd
\end{tcolorbox}

Esistono insiemi ordinati ma non ben ordinati? La risposta è affermativa:
\begin{itemize}
    \item $\mathbb{Z}$: il sottoinsieme $M\subseteq \mathbb{Z}$ dei numeri negativi non ha minimo.
    \item $\mathbb{R}$: l'intervallo dei reali $(0,1)$ non ammette minimo.
\end{itemize}




\section{\underline{Il principio di induzione di seconda forma}}
\begin{tcolorbox}[title={Principio di induzione di seconda forma}]
Sia $\{P(n)\}_{n \in \mathbb{N}}$ una famiglia di proposizioni
indicizzate su $\mathbb{N}$ e si supponga che:
\begin{enumerate}
    \item $P(0)$ sia vera
    \item $\forall n > 0 \quad (P(k) \text{ vera } \forall k < n) \Longrightarrow P(n)$
\end{enumerate}
Allora $P(n)$ è vera $\forall n \in \mathbb{N}$.
$\\\\$
\emph{\textbf{Dimostrazione:}} Sia $A := \{n \in \mathbb{N} | P(n) \text{ falsa} \}$,
e si supponga per assurdo che $A \not = \varnothing$. Allora per
la proprietà di buo ordinamento di $\mathbb{N}$, $A$ ammette minimo $m$
(essendo $A\subseteq\mathbb{N}$ non vuoto). Vale:
\[ m := \min A \Longrightarrow m \in A \Longrightarrow P(m) \text{ falsa } \Longrightarrow m \not = 0 \]
che segue da (1).

Se $k \in \mathbb{N}: 0 \leq k < m = \min A$ allora $k \not \in A$. Ma
allora $P(k)$ è vera. Per (2), segue che $P(m)$ è vera, ma si ottiene
così una contraddizione. Deve dunque essere necessariamente $A = \varnothing$,
ovvero $P(n)$ vera $\forall n \in \mathbb{N}$.
\cvd
\end{tcolorbox}