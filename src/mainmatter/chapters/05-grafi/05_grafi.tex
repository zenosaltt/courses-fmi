\chptr{Introduzione alla Teoria dei Grafi}

\section{Definizioni preliminari}
Non introdurremo i grafi nella classica maniera grafica e intuitiva
(punti collegati tra loro da semirette), ma sfrutteremo il linguaggio
insiemistico precedentemente studiato al fine di comprendere le proprietà
e le potenzialità di questi oggetti matematici, estremamente utili in
informatica.

\begin{tcolorbox}[colback=yellow!30, colframe=yellow!30!black, title=Insieme dei 2-sottoinsiemi]
Dato un insieme $V$, indichiamo con
\[ \binom{V}{2}:=\{ A \in 2^V| \left|A\right|=2 \} \]
(che si dice $V$ su 2) l'insieme i cui elementi sono tutti
i sottoinsiemi con due elementi che si possono estrarre da $V$.
\end{tcolorbox}



\section{Morfismi e isomorfismi}